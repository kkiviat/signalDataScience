% Created 2017-10-24 Tue 12:51
\documentclass[11pt]{article}
\usepackage[utf8]{inputenc}
\usepackage[T1]{fontenc}
\usepackage{fixltx2e}
\usepackage{graphicx}
\usepackage{longtable}
\usepackage{float}
\usepackage{wrapfig}
\usepackage{rotating}
\usepackage[normalem]{ulem}
\usepackage{amsmath}
\usepackage{textcomp}
\usepackage{marvosym}
\usepackage{wasysym}
\usepackage{amssymb}
\usepackage{hyperref}
\tolerance=1000
\usepackage[margin=2cm]{geometry}
\author{Kira}
\date{\today}
\title{sql-intermediate}
\hypersetup{
  pdfkeywords={},
  pdfsubject={},
  pdfcreator={Emacs 25.2.1 (Org mode 8.2.10)}}
\begin{document}

\maketitle
\tableofcontents

\section{Company data}
\label{sec-1}
\subsection{Get the data}
\label{sec-1-1}
\begin{verbatim}
sqlite3 data/company.db < data/company.sql
\end{verbatim}

\subsection{Set db file}
\label{sec-1-2}
Now using :db (print db) in the header of sqlite code blocks will set this as the database
\begin{verbatim}
(setq db "data/company.db")
\end{verbatim}

\subsection{Output the names of employees that Oliver Warbucks directly supervises}
\label{sec-1-3}
\begin{verbatim}
SELECT Name FROM records 
WHERE Supervisor='Oliver Warbucks';
\end{verbatim}

\subsection{Output all information about self-supervising employees}
\label{sec-1-4}
\begin{verbatim}
SELECT * FROM records WHERE Name=Supervisor
\end{verbatim}

\subsection{Output names of all employees with salary greater than 50000 in alphabetical order}
\label{sec-1-5}
\begin{verbatim}
SELECT Name FROM records WHERE Salary > 50000 ORDER BY Name;
\end{verbatim}

\subsection{Output table with columns: employee, salary, supervisor, supervisor's salary, containing all supervisors who earn more than twice as much as the employee}
\label{sec-1-6}
\begin{verbatim}
SELECT e.Name, e.Salary, e.Supervisor, s.Salary
FROM records e JOIN records s ON e.Supervisor=s.Name
WHERE s.Salary > 2*e.Salary;
\end{verbatim}

\subsection{Output names of employees whose supervisor is in a different division}
\label{sec-1-7}
\begin{verbatim}
SELECT e.Name
FROM records e JOIN records s ON e.Supervisor=s.Name
WHERE e.Division <> s.Division;
\end{verbatim}

\subsection{Output meeting days and times of all employees directly supervised by Oliver Warbucks}
\label{sec-1-8}
\begin{verbatim}
SELECT Day, Time FROM meetings
WHERE Division IN (SELECT Division FROM records WHERE Supervisor='Oliver Warbucks');
\end{verbatim}

or to attach names to the meetings:
\begin{verbatim}
SELECT Name, Day, Time
FROM records r JOIN meetings m ON r.Division=m.Division
WHERE Supervisor='Oliver Warbucks';
\end{verbatim}

\subsection{Output all names of middle managers (people who supervise someone and are supervised)}
\label{sec-1-9}
\begin{verbatim}
SELECT Name FROM records
WHERE Name != Supervisor
  AND Name IN (SELECT Supervisor FROM records);
\end{verbatim}

\subsection{Get names of all employees who have a meeting on the same day as their supervisor}
\label{sec-1-10}
\begin{verbatim}
SELECT DISTINCT e.Name
FROM records e JOIN records s ON e.supervisor=s.Name
	       JOIN meetings me ON e.Division=me.Division
	       JOIN meetings ms ON s.Division=ms.Division
WHERE me.Day=ms.Day;
\end{verbatim}

\subsection{Output each supervisor and the sum of salaries of all of each supervisor's employees}
\label{sec-1-11}
\begin{verbatim}
SELECT Supervisor, SUM(Salary) FROM records
GROUP BY Supervisor;
\end{verbatim}

\subsection{Output all salaries that appear more than once in the employee records}
\label{sec-1-12}
\begin{verbatim}
SELECT Salary FROM records
GROUP BY Salary
HAVING COUNT(*) > 1;
\end{verbatim}

\section{Recursive WITH statements}
\label{sec-2}
\subsection{Print out the factorial of the numbers 0 to 10}
\label{sec-2-1}
\begin{verbatim}
WITH num(n, f) AS (
SELECT 0, 1 UNION
SELECT n+1, (n+1)*f FROM num WHERE n <= 10
)
SELECT * FROM num;
\end{verbatim}

\subsection{Group natural numbers in 3-number-long sequences}
\label{sec-2-2}
Should generate this output:
\begin{center}
\begin{tabular}{rrr}
0 & 1 & 2\\
3 & 4 & 5\\
6 & 7 & 8\\
9 & 10 & 11\\
12 & 13 & 14\\
\end{tabular}
\end{center}
\begin{verbatim}
WITH num(a, b, c) AS (
SELECT 0, 1, 2 UNION
SELECT a+3, b+3, c+3 FROM num WHERE a <= 9
)
SELECT * FROM num;
\end{verbatim}

\section{Presidential dogs}
\label{sec-3}
\subsection{Get the data}
\label{sec-3-1}
\begin{verbatim}
sqlite3 data/dogs.db < data/dogs.sql
\end{verbatim}

\subsection{Get names of all dogs with a parent, ordered by height of the parent from tallest to shortest}
\label{sec-3-2}
\begin{verbatim}
SELECT d.name
FROM dogs d JOIN parents ON d.name=parents.child
	    JOIN dogs p ON parents.parent=p.name
ORDER BY p.height DESC;
\end{verbatim}

\subsection{Create a single string for every pair of siblings that have the same size, where each value is a sentence describing the siblings by their size, and each sibling pair only appears once in alphabetical order}
\label{sec-3-3}
Expected output:
\texttt{barack and clinton are standard siblings}
\texttt{abraham and grover are toy siblings}
\begin{verbatim}
WITH siblings AS (
     SELECT x.child AS name1, d1.height AS height1, y.child AS name2, d2.height AS height2
     FROM parents x JOIN parents y ON x.parent=y.parent
		    JOIN dogs d1 ON x.child=d1.name
		    JOIN dogs d2 ON y.child=d2.name
     WHERE name1 < name2
)
SELECT name1 || " and " || name2 || " are " || s1.size || " siblings"
FROM siblings AS s JOIN sizes s1 ON (s.height1 > s1.min AND s.height1 <= s1.max)
		   JOIN sizes s2 ON (s.height2 > s2.min AND s.height2 <= s2.max)
WHERE s1.size = s2.size;
\end{verbatim}

\subsection{Create a table describing all stacks of 4 dogs at least 170 cm high}
\label{sec-3-4}
The first column should contain a comma-separated list of dogs in the stack, and the second column should contain the total height of the stack. Order the stacks in increasing order of total height. A valid stack of dogs includes each dog only once, and the dogs should be listed in increasing order of height within the stack. \textbf{Assume no two dogs have the same height.}

Expected output:
\texttt{abraham, delano, clinton, barack|171}
\texttt{grover, delano, clinton, barack|173}
\texttt{herbert, delano, clinton, barack|176}
\texttt{fillmore, delano, clinton, barack|177}
\texttt{eisenhower, delano, clinton, barack|180}

\begin{verbatim}
WITH stacks4 AS (
     SELECT d1.name||", "||
	    d2.name||", "||
	    d3.name||", "||
	    d4.name AS list, d1.height+d2.height+d3.height+d4.height AS height
     FROM dogs d1 JOIN dogs d2 JOIN dogs d3 JOIN dogs d4
     WHERE d1.height < d2.height
     AND d2.height < d3.height
     AND d3.height < d4.height
)
SELECT * FROM stacks4
WHERE height >= 170
ORDER BY height ASC;
\end{verbatim}

or without a WITH:
\begin{verbatim}
SELECT d1.name||", "||
       d2.name||", "||
       d3.name||", "||
       d4.name AS list,                 d1.height+d2.height+d3.height+d4.height AS stack_height
FROM dogs d1 JOIN dogs d2 ON d1.height < d2.height
	     JOIN dogs d3 ON d2.height < d3.height
	     JOIN dogs d4 ON d3.height < d4.height
WHERE stack_height >= 170
ORDER BY stack_height ASC;
\end{verbatim}

or with a recursive solution:
\begin{verbatim}
WITH stacks(list, topdogHeight, height, n) AS (
     SELECT name, height, height, 1 FROM dogs UNION
     SELECT s.list||", "||d.name, d.height, s.height+d.height, n+1
     FROM stacks s JOIN dogs d ON (s.topdogHeight < d.height)
     WHERE n < 4
)
SELECT list, height FROM stacks WHERE n = 4 AND height >= 170 ORDER BY height ASC;
\end{verbatim}

\subsection{Get all pairs forming non-parent relations ordered by height difference, with shortest paired with tallest first, and tallest paired with shortest last.}
\label{sec-3-5}
\begin{verbatim}
WITH relation_pairs(d1, d2, dist) AS (
     SELECT parent, child, 1 FROM parents UNION
     SELECT parent, d2, dist+1
     FROM relation_pairs JOIN parents ON child=d1
),
all_relation_pairs(d1, d2, dist) AS (
     SELECT d1, d2, dist FROM relation_pairs UNION
     SELECT d2, d1, dist FROM relation_pairs
)
SELECT d1, d2
FROM all_relation_pairs JOIN dogs x ON d1=x.name
			JOIN dogs y ON d2=y.name
WHERE dist > 1
ORDER BY x.height-y.height;
\end{verbatim}
% Emacs 25.2.1 (Org mode 8.2.10)
\end{document}